%!TEX root = ../dissertation.tex
% the acknowledgments section

I'd like to start by thanking all the members of the rubidium lab that I have worked with. Any of results presented in this thesis would not be possible without their help and support. I also would like to thank my advisor Professor Markus Greiner for providing such a wonderful atmosphere in the group. Markus always welcomes his students' ideas about the directions for the experiments and provides guidance when we need it. He is very knowledgeable and technically gifted especially when it comes to optics. It was a great pleasure to work together and learn from him.

When I just started in the lab, I worked with our postdoc at that time Rajibul Islam. Together we learned a lot about aberrations and built a few optical setups. During first three years Alex Ma and Philipp Preiss they taught me all tricks they knew about the experiment. I'd like to thank them for enduring my constant questions and explaining me things in great details. It was a  lot of fun working with you guys! In the beginning of my fourth yeah Adam Kaufman joined the lab as a new postdoc. He immediately brought a new vibe to the team. I'm grateful to him for his efforts to improve me both personally and professionally.

During the last three years I worked very closely with Eric Tai, Matthew Rispoli and Robert Schittko. I think they all deserve an extra credit for enduring me over those years. I very much enjoyed working with Eric during the chiral orbits time, he was always motivating me to do things properly and pay attention to the details. We also had a lot of fun moment in the office during his last few years in the lab. I appreciate his patience for my endless questions about coding and spelling. Matthew deserves special credit as he was one of the driving forces behind our latest experiments. Hes diligence perfectly balanced my desire to move as fast as possible and his positive attitude, even during the time when stuff was not working for months, kept me going. Robert and his creative abilities made all our figures look awesome. I learned a lot about design and the art of presentation from him.

I'd like to acknowledge Soonwon Choi my officemate during our first year. We spent countless hours together doing homeworks and talking about physics. Soonwon taught me how to do numerical simulations and clarified a lot of concepts that I was confused about.

Lead by our new postdoc Julian L\'eonard young PhD student Sooshin Kim and Joyce Kwan are slowly taking over the experiment they are very enthusiastic and I'm sure they will find a lot of new and fun stuff to work on. I would like to specially acknowledge Julian for helping me with writing the paper. I think my writing skills have been greatly improved thanks to him. 

I also would like to thank the other members of the Greiner from lithium and erbium experiments. We always had fun discussions during lunch and had a great time on a group outings. I would like to highlight Sebastian Blatt and Anton Mazurenko for entroducing me to the field of low noise electronics. Special thanks to Julian, Robert, Matt and Geoffrey for reading and correcting various parts of this thesis.

I'm very grateful to Lisa Cacciabaudo and Carol Davis for making physics department my home. They always had a word of support when I needed it.I'd like to thank all my friend outside the lab who made my years at Harvard full of memorable moments. It was a great pleasure to be among all of you.

My family played a crucial role in making this happen. I'd like to thank my mom Galina and dad Alexander who never lost hope in me and always supported me when I needed it. I also would like to acknowledge my uncle Misha who provided a guidance and advice along the way. Lastly, I'd like to thank Theresa Rimmele for supporting me over the last two years. She deserves a special credit during the last few month of thesis writing. She was always there for me when I needed it.
