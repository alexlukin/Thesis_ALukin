%!TEX root = ../dissertation.tex
\chapter{Introduction}
\label{introduction}

The concept of thermalization is very familiar to every one of us from the everyday life. When two systems are brought in contact with each other and are allowed to exchange the energy and particles, they come to an equilibrium state. When the number of degrees of freedom in one system is substantially larger than in the other one, its state cannot be sufficiently altered due to the coupling between the two. In this case, we call a larger one bath. Coupling to a bath drives the system into thermal equilibrium, such that its state can be described by the thermodynamic ensemble. This presents a very powerful theoretical tool since the complicated microscopic dynamics can be described by a small number of thermodynamic variables.

Seemingly, this concept does not apply to the pure quantum states, as they are completely isolated from the environment. During the time evolution, the purity of the state does not change even if the system's parameters are suddenly changed. However, in the wide class of systems this coherent many-body dynamic dives local thermalization of the subsystem. In this case, any local observable appears thermal even though the full state remains pure \cite{Deutsch1991, Olshanii2008,Eisert2015}. Strikingly, even with infinitely many copies of such quantum state, the local observables are fundamentally unable to reveal whether this is a single quantum state or a thermal ensemble. In other words, a globally-pure quantum state is apparently indistinguishable from a mixed, globally-entropic thermal ensemble~\cite{Shankar1985, Deutsch1991, SrendickiETH, Olshanii2008}.

This leads us to an interesting thought: the world around us is very well described by the classical thermodynamics, which would suggest that the universe as a whole is one giant mixed state rather than in a pure quantum state. However, since there is no way of locally distinguishing between the two, we might never be able to find an answer to this conundrum, unless we happen to find an identical copy of our universe and perform the measurement of purity similar to those explained in this thesis.

In the classical systems, the thermalization is justified through the notion of ergodicity that states that the system explores the entire parameter space allowed by the constraints. This leads to the concept of entropy maximization. In the quantum setting the role of classical entropy is taken by the entanglement: Thermalizing systems develop an extensive amount of entanglement between the subsystems, therefore ignoring the information about the rest of the system puts the subsystem in the statistical mixture of a large number of states. From this perspective, we can say that thermalizing quantum systems become an effective bath to themselves.

A more formal explanation of the quantum thermalization is given by the Eigenstate Thermalization Hypothesis, which connects it with the properties of the many-body eigenstates of the system. Although being numerically verified for in the number of systems \cite{ Olshanii2008, Phys. Rev. E 86, 010102(R), DAlessio2015, Phys. Rev. B 91, 155123}, the full theoretical proof of ETH remains elusive to this day. Therefore, experimental studies of such this phenomenon are of a great interest in the field.

Recent experiments have demonstrated analogies between classical chaotic dynamics and the role of entanglement in few-qubit spin systems~\cite{Martinis2016}, as well as the dynamics of thermalization within an ion system~\cite{Schaetz2015}. Furthermore, studies of bulk gases have shown the emergence of thermal ensembles and the effects of conserved quantities in isolated quantum systems through macroscopic observables and correlation functions~\cite{Trotzky2012,Langen2013,Geiger2014,Langen2015}. However, those experiments were not able to directly verify the purity of the full state -- key feature distinguishing quantum thermalization form the coupling to the external classical bath.

An exception to the paradigm of quantum thermalization can be found in the integrable systems \cite{ Phys. Rev. Lett. 100, 210403 (2008)}, where fine-tuning of the parameters results in the emergence of an extensive number of the conserved quantities even at the local level. However, small deviations from the fine-tuned point immediately restore the system’s ergodicity \cite{ Phys. Rev. X 8, 021030}.

The only known robust example of thermalization breakdown is provided in the disordered systems, which localizes the particles\cite{ Anderson1958}, leading to suppression of the transport through the system. This prevents the buildup of entanglement in the system, which is crucial for thermalization. Surprisingly, the localization persists even in the presence of interactions between the particles, this phenomenon is called many-body localization (MBL) \cite{ Anderson1958, Gornyi2005, Basko2006, Oganesyan2007}.

Since in the MBL state the particles are localized one might expect, that there would be no dynamics in such a system, which turns out to be true for local observables. However, numerical studies have shown that entanglement still builds up in the system \cite{Znidaric2008, Bardarson2012}, although it happens over exponentially long timescales. This happens due to the presence of interactions, which give rise to slow coherent many-body dynamics that generate non-local correlations, inaccessible to local observables \cite{Serbyn2013, Serbyn2013a, Huse2014}. These dynamics are considered to be the hallmark of MBL that distinguishes it from its non-interacting counterpart, called Anderson localization \cite{Anderson1958}.

Experimental studies have identified MBL through the persistence of the initial density distribution \cite{Schreiber2015, Smith2015, Bordia2016, Choi2016, Lueschen2017, Bordia2017} and two-point correlation functions during transient dynamics \cite{Smith2015}. However, the observation of the buildup of non-local correlations across the full system has remained elusive, because it requires exquisite control over the system's coherence over exponentially long evolution time.

In this thesis, I present experimental techniques that extend the capabilities of our quantum gas microscope \cite{Bakr2009} to explore the coherent many-body dynamics of the one-dimensional Bose-Hubbard system over multiple orders of time evolution, exceeding the previously achieved timescales by more than an order of magnitude \cite{Smith2015}. Using holographic beam shaping we achieve precise control over optical potentials at a single site level in our system. The ability to address individual sites enables a deterministic preparation of desired initial states as well as the ability to control the potential landscape of the system during the time evolution.

The advanced capabilities of our quantum gas microscope enable collective measurement on multiple copies of the system, which enables the measurement of purity of a quantum state and. It also gives us an access to the entanglement entropy between the subsystems of the quantum state. The ability to control optical potential with high accuracy on a single-site level enables us to engineer custom potential landscapes in the system.

A quick summary of the experiments presented in this work is provided as follows, according to the organization of the rest of the thesis:
\begin{description}
	\item[$\bullet$ Chapter 1] gives the overview of the experimental setup, the quantum gas microscope. We highlight all parts of the setup that would be important for the future discussions in this thesis, and summarize the important concepts required for understanding the theory of cold-atom experiments in the optical lattices.
	\item[$\bullet$ Chapter 2] describes how we use digital micro-mirror device (DMD) in the Fourier plane of our imaging system in order to create arbitrary potential landscapes on the atoms. We summarize the calibration procedure that allows us to achieve a diffraction-limited performance on the atoms. We aslo discuss the limits of this technique related to the hologram binarization.
	\item[$\bullet$ Chapter 3] shows how we achieve initial state preparation with near-perfect fidelity, using single site control described in the previous chapter. We discuss one of the decoherence effects coming from the atom number loss in our system, and describe how we can mitigate it by using full number-resolved readout of the final state, developed in this chapter.
	\item[$\bullet$ Chapter 4] focuses on characterization of the coherence in our system. We introduce the concept of many-body interference using a beamsplitter technique in order to measure the purity of the quantum state. Using this technique we demonstrate the quantum coherence in $\sim12000$ dimensional Hilbert space over multiple orders of time evolution. We also introduce the measurements of coherent quantum evolution with a single copy, that allows us to extend our characterization techniques beyond the two copy method. 
	\item[$\bullet$ Chapter 5] describes the experiments, where we observe how a small subsystem of a globally pure state exhibit thermalization during the time evolution of the system. We measure the entanglement entropy of the subsystems of the thermalization state, and determine the correspondence of the quantum entanglement entropy between the subsystems of the pure state and the classical thermal entropy%\cite{Rigol2012, Deutsch2013}.,
	 which explains the microscopic origins of thermalization in the system.
	\item[$\bullet$ Chapter 6] presents the experiments with disordered system. We realize a one-dimensional many-body localized system and characterize its key properties: breakdown of thermalization, spatial localization, and study it's entanglement properties over multiple decades of time evolution. In order to distill the quantum phenomenon in the system we introduce and separately measure two contributions to the entanglement entropy corresponding to a particle transport and quantum phase evolutions in the system.
	 \item[$\bullet$ Chapter 7] concludes with the summery of the work done in this thesis.
\end{description}