%!TEX root = ../dissertation.tex

\chapter{Entanglement and it's measures}

\section{Entanglement in a quantum system}
Quantum mechanics gives us tools to study highly correlated system. The most famous example is an entangled Bell state \cite{something}, which can be written in the context of two spin $\frac{1}{2}$ particles as 
\begin{equation}
\left| \psi \right>=(\left| \uparrow \downarrow \right>+ \left| \downarrow \uparrow \right>)/\sqrt{2},
\end{equation}
where up and down arrows represent one of the two states of each spin. The remarkable property of this state is, that no matter which basis one would choose to measure both particles in, the results of the measurement will always be anti-correlated. This entanglement could further be utilized as a resource for a variety of application ranging from metrology and sensing \cite{sombody} to quantum computing \cite{Chuang book} and secure quantum communications \cite{something}. However those correlations are very fragile: addition of any amount of classical mixture of $\left| \uparrow \right>$ and $\left| \downarrow \right>$ states to any of the spins immediately destroys the perfect correlations between the two, that ultimately limits the practical applications of quantum techniques. In this chapter we will show, how using the single site control over the atoms in our quantum gas microscope, we can prepare and control highly entangled many-particle quantum systems, in particular in the context of quantum simulations. 