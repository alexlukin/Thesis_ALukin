%!TEX root = ../dissertation.tex
%\begin{savequote}[75mm]
%Nulla facilisi. In vel sem. Morbi id urna in diam dignissim feugiat. Proin molestie tortor eu velit. Aliquam erat volutpat. Nullam ultrices, diam tempus vulputate egestas, eros pede varius leo.
%\qauthor{Quoteauthor Lastname}
%\end{savequote}

\chapter{Holographic approach to arbitrary potential generation}

\section{Why do we need custom optical potentials?}
The main advantage of cold atoms experiments is an ability to prepare and probe the system on characteristic time and length scales (for a lattice models those would be a tunneling time and lattice spacing respectively). Quantum gas microscope enables us to observe our lattice system with a single site resolution, which lead to the first direct observation of superfluid to Mott insulator \cite{Bakr2010, bloch} as well as paramagnet to aniferromagent \cite{Simon2011, Parsons??} quantum phase transitions. However, in order to fully enable the capabilities of such quantum simulators one needs a tool to arbitrary control the shape of optical potentials in the system. This would enable initialization of initial states with high fidelity \cite{us, somebody else?}, studies of the systems with complex \cite{Tilman QPC, Roatti QPC} and dynamically changing \cite{??} geometries. To achieve this, one has to control the light intensity on $\approx 1$ lattice site scale, this can be achieved by means of spatial light modulators (SLM).

\section{Spatial light modulators in the image plane}
The first straight forward way is to use the microscope in reverse, by imaging a desired light pattern onto the atoms \cite{P. Schauß thesis, RMA thesis, MAZU thesis}. Although this method is very powerful for creating large scale, smooth-varying potentials, it has two major challenges. \textit{First}, the wavelength of conservative light potentials, that one wants to project, is generally very different from the atom imaging wavelength, which most quantum gas microscopes are optimized for. This can lead to significant distortion of the desired wavefront due to chromatic aberrations, if the imaging system is not specifically corrected for that. First two leading orders of aberrations are given by defocus and spherical aberration. This problem can be solved by either using an achromatic imaging system, that is wavelength insensitive in the desired range, or precompensating the imaged wavefront for the defocus cosed by the imaging. \textit{Second}, it is fundamentally impossible to create $100 \%$ intensity modulation of the potential on the single site scale without using Fourier filtering and phase control of the wave front. 

To illustrate the last point, let's consider a single lens with a fixed aperture of size $d$ in the Fourier plane and focal length $f$ (see fig~\ref{fig:DMD_lens}). To keep the math simple let's consider only two spatial dimensions $x$ and $z$. The sharpest feature that one can make in the image plane with constant phase wavefront in the Fourier plane is so called point spread function, with the resulting light intensity in the image plane given by $I(x) \sim sinc^2(\frac{\pi}{\Theta} x)$, where $\Theta = \frac{\lambda f}{d}$ (see fig~\ref{fig:DMD_lens} A). The question becomes: how can one make an intensity pattern that is periodic with period $\Theta$? There are two different electric field profiles that would give us the same intensity profile satisfying the constraint: $E \sim \sqrt{(1+cos(\frac{\pi}{\Theta}x))/2}$ and $E \sim cos(\frac{\pi}{2\Theta} x)$, both yielding $I \sim (1+cos(\frac{\pi}{\Theta}x))/2$ (see fig~\ref{fig:DMD_lens} B,C). The first one has only positive electric field components and hence the same phase, however the Fourier transform of it occupies the Fourier plane of twice the size of our initial point spread function. The second one has the same Fourier plane size as the one we started with at the expense of having electric field become negative or alternate the phase between $0$ and $\pi$ between adjacent maxima.
%The characteristic size of the feature is $FWHM = 0.84 \Theta$.

\section{Spatial light modulator in the Fourier plane}
Although, phase control of the electric field using image plane DMD can be achieved using number of phase coding methods along side with spatial filtering in the Fourier plane \cite{Lee1970, Goorden2014}, one still needs to correct for the aberrations, the beam encounters on its way form the SLM to the desired image plane (in our case we call it atom plane). Note, that one needs to correct for both intensity pattern, that eliminates the SLM, as well as any phase deviation from the desired wavefront, which might be particularly challenging for imaging systems with high numerical aperture, due to the breakdown of the paraxial approximation. In order to achieve the best possible quality of the optical potentials it would be desirable to be able to actively measure and compensate for aberration in the system, here the use of the SLM in the Fourier plane becomes particularly useful.

There are two main types of SLMs that are being used in such setups: liquid-crystal display (LED) based and digital micro-mirror device (DMD) based. LEDs can be configured to provide amplitude modulation or phase modulation of the incident beam, using the birefringence of liquid crystals. With optimization and feedback algorithms [83, 84], it is possible to create complex, re-programmable potentials, using this technology. However this type of SLMs have one major drawback when it comes to cold atoms experiments. Due to the nature of liquid crystals, they have to be placed in the into the switching electric filed, that oscillates in $kHz$ range, causing the output light to blink at the same frequency. This blinking could potentially be a problem, since the typical trap frequencies of the atoms in optical lattice experiments lay in the same frequency range, and hence the blinking light might induce undesired heating in the system

.

 

