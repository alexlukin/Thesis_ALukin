%!TEX root = ../dissertation.tex
\chapter{Introduction}
\label{introduction}

Studies of quantum many-body systems attract a lot of attention due to the reach amount of exciting phenomenon that can be observed in them. The examples range from high-temperature superconductivity \cite{Rev. Mod. Phys. 78, 17 (2006)} and topological fractional quantum Hall states \cite{Rev. Mod. Phys. 71, S298 (1999)} to magnetism in spin systems \cite{. Auerbach, Interacting Electrons and Quantum Magnetism (Springer, 1994)}. These effects stem from strong interactions between constituent particles that lead to non-classical correlations among them.

Both experimental and theoretical work has been done, in order to achieve a better understanding and fully harness all properties of quantum systems. A lot of progress has been made in understanding the properties of the ground states in different systems, which led to the iconic experimental observations of various new phases of matter. Among the examples are Bose-Einstein condensates \cite{wolfang, cornell}, bosonic \cite{Greiner2002} and fermionic \cite{tilman} Mott insulators and quantum Hall states \cite{quantum hall}. 

The developments in both numerical and experimental capabilities sparked a new interest in non-equilibrium phenomena, as it generally requires a full quantum treatment of the entire system. Latest experimental results have already deepened our understanding of transport in out-of-equilibrium setting \cite{tilman}, as well as lead to the discovery of the novel dynamical phases of matter such as time crystals \cite{mich monroe}.

Theoretical studies of complex quantum systems can be broadly divided into two different classes. One is to come up with a simplified effective model that adequately captures the properties of the system of interest and the other is to use numerical tools in order to find the exact solutions to the quantum mechanical equations of motion. Despite a great success in the past\cite{Gross pitaevski, femi liquid}, the former is becoming less and less accurate as a system's complexity increases, since such systems develop a large amount of non-local quantum correlations between their constituents, that can not be efficiently captured.Therefore, an increasing number of new phenomenon require the full quantum-mechanical treatment. However, it has one major problem: the computing power required to simulate a system increases exponentially with its size. This is due to the property of quantum mechanics, which allows the system to occupy multiple states at the same time, and, generally, the more complex the system is the more classical states it occupies.

One of the possible solutions to this problem was proposed by Feynman \cite{nt. J. Theor. Phys. 21,467 (1982)} called quantum simulations. The idea is to build an easy to control and probe quantum system that is governed by the Hamiltonian of interest. By making a measurement of different observables one can then determine various correlations and properties of the model. Up to date there exist a number of different platforms that realize such a setting including trapped ions \cite{Nature Phys.8, 277 (2012)}, superconducting circuits \cite{Nature Phys. 8, 292 (2012)} and cold atoms \cite{Nature Phys. 8, 267 (2012)}.

Cold atoms are a particularly attractive platform as they offers a number of useful feathers: A variety of both bosonic and fermionic atoms can be cooled to the regime of quantum degeneracy. The contact interactions between the atoms can be tuned via so-called Feshbach resonances \cite{Rev. Mod. Phys., 82:1225, 2010}. In addition, recently a number groups were able to cool the atoms with sufficiently large magnetic-dipole moment \cite{Phys. Rev. Lett. 107, 190401,  Phys. Rev. Lett. 108, 210401}, that allows for tunable long-range interactions in the system. Finally, the ability to create conservative potentials using light allows for easy manipulation and trapping.  

Combined with optical lattices cold atoms offer a robust platform for the study of lattice systems, as they offer a large tenability of key parameters such as characteristic time scales via lattice spacing, and geometry by altering the laser configuration. The development of microscopy techniques \cite{Bakr2009, Nature 467, 68 (2010)}, in addition, allows to manipulate and read out the system on a single site level giving a direct access to multi-point correlation functions in the system.

The main challenge for every quantum simulator is to ensure the coherent evolution of the system over a long time.
It becomes especially challenging in the large systems, since the decoherence mechanisms typically scale exponentially as the particle number is increased. Up to date, the coherence of itinerant quantum systems was limited to $\sim10$ tunneling times\cite{Smith2016}. In turn, this sets the limit on size of the systems one can study in such settings.

In this thesis, I present experimental techniques that extend the capabilities of quantum gas microscopes to explore the coherent many-body dynamics of the one-dimensional Bose-Hubbard system over multiple orders of time evolution up to $\sim 100$ tunneling times. Using holographic beam shaping we achieve precise control over optical potential at a single site level in our system. The ability to address individual sites enables a deterministic preparation of desired initial states as well as the ability to control the potential landscape of the system during the time evolution.

Despite the increasing complexity of out-of-equilibrium systems, numerical studies have revealed one surprising fetcher: an isolated quantum system appears to thermalize under its own dynamics unaided by a reservoir~\cite{Deutsch1991, Olshanii2008,Eisert2015}, so that the tools of statistical mechanics apply and challenging simulations are no longer required. In this case, a quantum state, coherently evolving according to the Schr\"{o}dinger equation, is such that local observables can be predicted from a thermal ensemble and thermodynamic quantities. 

Strikingly, even with infinitely many copies of this quantum state, the local observables are fundamentally unable to reveal whether this is a single quantum state or a thermal ensemble. In other words, a globally-pure quantum state is apparently indistinguishable from a mixed, globally-entropic thermal ensemble~\cite{Shankar1985, Deutsch1991, SrendickiETH, Olshanii2008}. However, the verification of such prediction is experimentally challenging, since we need a tool to distinguish coherent many-body quantum states form the globally thermal ensembles.

This leads us to an interesting thought, as the reader might know, the world around us is very well described by the classical thermodynamics, which would suggest that the universe as a whole is one giant mixed state rather than in a pure quantum state. However, since there is no way of locally distinguishing between the two, we might never be able to find an answer to this conundrum, unless we happen to find an identical copy of our universe and perform the measurement of purity similar to those explained in this thesis. 

The advanced capabilities of quantum gas microscopes enable collective measurement on multiple copies of the system, which allows us to measure the purity of a quantum state. It allows us to observe how a small subsystem of a globally pure state exhibit thermalization during the time evolution of the system. Also, we determine the correspondence of the quantum entanglement entropy between the subsystems of the state and the classical thermal entropy\cite{Rigol2012, Deutsch2013}., which explains the microscopic origins of thermalization in the system.

Although thermalization is inherent to almost all know quantum systems, an exception to this paradigm is provided in disordered systems. Presence of disorder localizes the particles\cite{ Anderson1958}, leading to suppression of the transport through the system. This prevents the buildup of entanglement in the system, which is crucial for thermalization. Surprisingly, the localization persists even in the presence of interactions between the particles, this phenomenon is called many-body localization (MBL) \cite{ Anderson1958, Gornyi2005, Basko2006, Oganesyan2007}.

Since in the MBL state the particles are localized one might expect, that there would be no dynamics in such a system, which turns out to be true for local observables. However, numerical studies have shown that entanglement still builds up in the system \cite{Znidaric2008, Bardarson2012}, although it happens over exponentially long timescales. 

This effect happens due to the presence of interactions, which give rise to slow coherent many-body dynamics that generate non-local correlations, inaccessible to local observables \cite{Serbyn2013, Serbyn2013a, Huse2014}. These dynamics are considered to be the hallmark of MBL and distinguish it from its non-interacting counterpart, called Anderson localization \cite{Anderson1958, Schwartz2007, Billy2008, Roati2008, Lahini2008, Kondov2011, Jendrzejewski2012, Semeghini2015}. Their observation, however, has remained elusive, because it requires exquisite control over the system's coherence and access to non-local observables.

The superb control over the system allows us to realize a one-dimensional MBL system and study it's entanglement properties over multiple decades of time evolution. In order to distil the quantum phenomenon in the system we separately measure two contributions to the entanglement entropy corresponding to a particle transport and quantum phase evolutions in the system. This allows us to unambiguously establish MBL as a distinct dynamic phase.