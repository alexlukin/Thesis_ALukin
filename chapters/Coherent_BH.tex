%!TEX root = ../dissertation.tex
\chapter{Long coherent dynamics in optical lattices}

\section{Law entropy initial states}
Quantum mechanics gives us tools to study highly correlated system. The most famous example is an entangled Bell state \cite{something}, which can be written in context of two spin $\frac{1}{2}$ particles as 
\begin{equation}
\left| \psi \right>=(\left| \uparrow \downarrow \right>+ \left| \downarrow \uparrow \right>)/2,
\end{equation}
where up and down arrows represent one of the two states of each spin. The remarkable property of this state is, that no matter which basis one would chose to measure both particles in, the results of the measurement will always be anti correlated. This entanglement could further be utilized as a resource for variety of application ranging form metrology and sensing \cite{sombody} to quantum computing \cite{Chuang book} and secure quantum communications \cite{something}. However those correlations are very fragile: addition of any amount of classical mixture of $\left| \uparrow \right>$ and $\left| \downarrow \right>$ states to any of the spins immediately destroys the perfect correlations between the two, that ultimately limits the practical applications of quantum techniques. In this chapter we will show, how using the single site control over the atoms in our quantum gas microscope, we can prepare and control highly entangled many-particle quantum systems, in particular in the context of quantum simulations. 

In our experiments the goal is to study the dynamics of strongly correlated many-body systems. Due to the probabilistic nature of quantum mechanics, in order to learn information about a quantum state, one needs to have an access to multiple copies of the same state. Then by doing repeated measurements, the probability distribution of occupying each particular basis state can be obtained. This means, that we also need a way to deterministically prepare the same initial state with high fidelity. 

In order to create an entangled state, one starts with a product state and then apply an appropriate entangling Hamiltonian to arrive at desired state.  For atoms in the optical lattices one of the examples of a conceptually simple initial state is a product state of given atom numbers on individual lattice sites. For all the experiments described bellow, we would like to prepare strings of single atoms of fixed length. There are to ways to achieve that: one is a bottom up approach, where by using single site addressing and control one can trap a single atom into a tightly focused optical tweezer. Then by using Raman cooling techniques the atom's vibrational state can be cooled to almost perfect ground state in all three dimensions \cite{adamo, selim}. Finally, by subsequent rearranging an array of such tweezer traps, in order to eliminate defects, one can achieve deterministic strings of up to 53 atoms \cite{misha} or $?x?$ two dimensional arrays with unity feeling \cite{broweys}. This technique has successfully enabled study of spin models, using Rydberg blockade techniques \cite{misha}. However, the current state of the art cooling fidelities do not allow for efficient preparation of single band Hubbard model with large number of particles, due to residual exited state fraction scaling exponentially with the number of atoms.

An alternative route is a top down approach, where one uses an ensemble of atoms in order to first crate a macroscopic number of atoms in the absolute ground state \cite{BEC, DFG}, then selectively isolate those, and finally adiabatically load them into an optical lattice \cite{Greiner2002}. Then, by increasing the onsight interaction strength between the atoms, by ramping up the lattice depth, one arrives at the Mott insulating state (see fig.~\ref{fig:CTE_MI}A), where each lattice site is occupied by an integer number of atoms according to a local chemical potential, with a vanishing atom number variance \cite{Bakr2010, Bloch MI}. Due to the finite ramp time between two many-body states it is impossible to remain exactly in the ground state of the system \cite{subir phase transition}. Therefor the resulting atom number distribution deviates from the ideal one by "smearing" the boundary between the shells with different occupation numbers (see fig~\ref{CTE_MI}B). However, if the Mott insulator is sufficiently large we can find a region with unity filling up to $12$ sites long. Hence, to achieve our desired target state, we just need to isolate such a region, and ensure that it doesn't get disturbed by the adjacent atoms.

\section{Cutting procedure}

In order to isolate a desired region of the lattice we preform so called "cutting" procedure. Using the DMD we project finite size lattice potential on top of initial $2D$ lattice, in which the Mott insulator was originally crated. Then by applying slow varying anticonfinement beam and switching off the the $2D$ lattice we make the atoms leave, except for those that were kept by the DMD potential (see fig.~\ref{fig:CTE_cutting}A). Finally we can reload the atom back into the $2D$ lattice by ramping it back up and switching off the DMD potential. By repeating this procedure along both directions, we managed to isolate plaquettes up to $12x2$ in size (see fig.~\ref{fig:CTE_cutting}B). This technique allows us to achieve $99\%$ single atom transfer fidelity form initial Mott insulator. Cutting more then $12$ sites along one direction using this method is limited by the available power of the DMD beam. Similar results have been achieved using state dependent light shifts and microwave pulses \cite{Bloch single site addresing}.

By increasing the DMD beam power we could in principal initialize even longer chains, however we have another difficulty in our system. In order to achieve high probability on $n=1$ shell we rely on the idea of entropy redistribution in our system \cite{some entropy redistribution}. It can be summarized in the following way: crossing the transition from Superfluid to Mott insulator inject the system with certain amount of excitations. If the system would ne completely homogenizes those excitations would spread evenly across the system. However, there is a strong inhomogeneity in the regions where shells of different occupation number touch one another. Since the density has to smoothly connect from one occupation to another, it deviates from an integer value creating regions of suprefluid. Superfluid has significantly smaller gap between law lying states compared to a Mott shell, hence it is energetically favorable for the excitations to concentrate in the regions between the shells. By thinking in terms of entropy per site, it is clear that this mechanism better for larger ration of the superfluid region to the Mott shells. This technique has alsp found a great success in fermi gas microscopes, where manual increase of the entropy reservoir region led to significant entropy reductions in the gapped part of the system \cite{Mazurenko2017, Chiu2018}.

 From the analysis above it fallows, that in order to create an singly occupied shell, it is advantageous to have it be surrounded between $n=0$ and $n=2$ shells. However, due to large amount of disorder in our lattice it is particularly challenging to create large, round Mott insulators, such that one can isolate a single line at unity filling longer then $\sim 16$ sites. This means that, in order to crate longer chains one has to switch to a different Mott insulator geometry to begin with. One simple solution that comes to mind is to make a rectangular box confinement instead of a harmonic one. In order to have well defined shells in this case one can use a magnetic field gradient in order to create a uniform tilt. We realize such configuration in our system in a $24$ site long box (see fig.~\ref{fig:CTE_MI_box}), then by using a cutting procedure along the other direction we were able to achieve $24x2$ plaquettes. In this case the length was only limited by the DMD beam power, that provided the confinement during the cutting procedure. By increasing the DMD power and reducing the disorder, coming from $2D$ lattice, this approach can enable the initialization of even longer chains of controlled length, which are particularly interesting for quantum simulations purposes.
 
 \section{Atom number resolved state readout}
 
 One of the drawbacks of quantum gas microscopes is so called parity projection, during the imaging process. When subject to a resonant light two atoms undergo photon assisted collisions, resulting in the molecule formation with a large kinetic energy. The molecules are not trapped by the optical lattice and get lost during this process. This means, that in our image the sites with two particles on the same site appear dark like an empty site and the sites with three particles appear as one. Although it is not a problem in the Mott insulator regime, it leads to the loss of large amounts of information about the state in the superfluid regime. 
 
 However, for a one dimensional system we can utilize the direction transverse to the chain, in order to achieve full number resolution. The idea behind this method is simple: once the dynamics along the chain is frozen (in our case we do this by rapidly increasing the lattice depth along the chain to $45 \textrm{Er}$), we can switch the confinement transverse to the chain (which is provided by the other optical lattice in this case) to spread the atoms from every site of the chain into tubes in transverse direction. In our experiment we can do such spreading over $\sim 200$ lattice sites, so that the probability of doubly occupying any site of the tube goes like
 \begin{equation}
 p(n,L) = ,
 \end{equation}
 where $n$ is the number of particles that start on the site in the beginning and $L$ is the length of the expansion tube. An example of such process is shown in figure \ref{fig:CTE_fullcount_one_line}.
 
 Although this technique is very powerful for isolated one dimensional systems, we would like to develop a method that would allow us to preform counting of a $2xN$ plaquettes. At first the solution seems very simple: one just needs a way to isolate the two sides of the plaquette and repeat the above procedure. Using the DMD we can project a single site wide Gaussian beam to disable the tunneling between two sides of the plaquette. However that leads to an issue: the potential required to completely suppress the hopping between two sides has sufficient hight at the position of the atoms, serving as an effective hill the atoms roll down from. This process imparts momentum onto the atom in the direction away from the barrier potential, which is large compared to the diffusion rate of the atoms relative to their center of mass. The result is that atoms can only spread across $\sim 15$ lattice sites, before they leave the field of view of the microscope. Such limited spread results in significant probability of double occupancies, when the initial number of atoms is above $3$, and hence not suited for our purposes.
 
 To mitigate this issue we add another step to this process. After a very short expansion time, when the edge of the atom distribution moves few sites away from the barrier, where it's effect can be neglected. We briffly capture the atoms back into the lattice and then release them again.