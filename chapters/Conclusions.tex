%!TEX root = ../dissertation.tex

\chapter{Conclusions and outlook}

In this thesis, I have presented techniques for preparation and readout of various number states in one-dimensional Bose-Hubbard model with high fidelity. This allows us to prepare highly pure initial states and study their coherent many-body dynamics over multiple orders of time evolution, more then an order of magnitude longer then has been previously demonstrated. 

Exquisite control over our system using digital micro-mirror device, that was pioneered in our group before, allowed us  to experimentally verify the purity of a quantum state and observer how its subsystems undergo quantum thermalization during the time evolution. By comparing various local observables to the predictions of thermodynamic ensembles we were able to verify the validity of eigenstate thermalization hypothesis in our system. By using a beamsplitter technique we studied the dynamics of the entanglement entropy in the thermalizing many-body system and made a connections between it and the classical thermodynamic entropy in the thermal state.

By adding disorder to the system we realized a many-body localized state -- the only know robust exception for thermalization in the quantum system. We observed the breakdown of eigenstate thermalization hypothesis by comparing the entropy of a single site density matrix to the predictions of thermal ensemble. We, for the first time, measured the correlation length of particles in the system with uniform density. Long coherent evolution in our experiment allowed us to observe the hallmark feature of such many-body localized state: the formation of non-local correlations in the absence of transport, qualitatively distinguishing it from non-interacting disordered systems.

In the future, the study of larger systems provide the way to probe the scaling predictions of eigenstate thermalization hypothesis. Conversely, the study of integrable Hamiltonians, where thermalization fails due to the emergence of an extensive number of conserved quantities as a result of fine-tuning of the system parameters, could provide a direct test between thermalization and conservation lows. The studies of fullcounting statistics after the quench will revival the relations between full quantum dynamics and various effective descriptions of the system.

In the context of thermal to many-body localized transition it would be interesting to study the behavior of the system at the intermediate values of the disorder depth. By varying the systems size the nature of the transition can be reviled. An alternative way to gain knowledge of the transition is to study a many-body localized system locally coupled to a thermal bath. The full control over the disorder potential makes our system a perfect candidate to carry out such experiments. 