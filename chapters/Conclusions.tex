%!TEX root = ../dissertation.tex

\chapter{Conclusions and outlook}

In this thesis, I have presented techniques that allow us to image a one-dimensional system of atoms in optical lattice with full number resolution on a single-site level. This extends the capabilities of quantum gas microscopes beyond parity-projection imaging, and together with the ability to prepare pure initial states with high fidelity enables us to study coherent many-body dynamics over multiple decades of time evolution, more than an order of magnitude longer than has been previously demonstrated

Exquisite control over our system using the digital micro-mirror device, that was pioneered in our group before, allowed us to experimentally verify the purity of a quantum state and observer how its subsystems undergo quantum thermalization during time evolution. By comparing various local observables to the predictions of thermodynamic ensembles, we were able to verify the validity of the eigenstate thermalization hypothesis in our system. Using a beamsplitter technique we studied the dynamics of entanglement entropy in a thermalizing many-body system and were able to connect it with the notion of classical thermodynamic entropy in a thermal state. These experiments pave the way for understanding how classical statistical mechanics emerges from the quantum many-body dynamics.

By adding disorder to the system, we realized a many-body localized state -- the only know robust exception to thermalization in a quantum system. We studied the breakdown of the eigenstate thermalization hypothesis by comparing the entropy of a single-site density matrix to the predictions of a thermal ensemble. For the first time, we have observed the localization of interacting particles in real space by measuring the density-density correlation length in the system with uniform density. Long coherent evolution in our experiment allowed us to observe the hallmark feature of such a many-body localized state: the formation of non-local correlations in the absence of transport, qualitatively distinguishing it from non-interacting disordered systems. We also verified that the state is truly long-range entangled by studying the scaling of the entropy as a function of the subsystem size.

In the future, the study of larger systems will provide a way to probe the scaling predictions of the eigenstate thermalization hypothesis. Conversely, the study of integrable Hamiltonians, where thermalization fails due to the emergence of an extensive number of conserved quantities as a result of fine-tuning of the system parameters, could provide a direct connection between thermalization and conservation laws. The studies of full-counting statistics after the quench will reveal the relationship between full quantum dynamics and various effective descriptions of the system.

In the context of thermal to many-body-localized transition, it is of great interest to understand the behaviour of the system at intermediate values of the disorder depth. By varying the size of the system, the nature of the transition can be revealed. An alternative way to gain knowledge of the transition is to study a many-body localized system locally coupled to a thermal bath. Full control over the disorder potential makes our system a perfect candidate to carry out such experiments.